\documentclass[main.tex]{subfiles}

\begin{document}
\subsection{Define the notion of time complexity \text{TIME}(f (n)). Explain why we can say that this is a worst-case measure of complexity ? Why do we use big $O$ notation when we reason on the time complexity of a Turing machine}

\subsection{Define the big $O$ notation and the small $o$ notation. Give examples and explain the relation that exists between those two notions}

\subsection{Why are deterministic polynomial time Turing machines suitable to study the class $P$}

\subsection{Define the running time of a Turing machine that is a decider. If $M$ is a nondeterministic machine that decides the language $L$ in $O(t(n))$, how can we bound the running time of a deterministic Turing machine that decides $L$}

\subsection{Define the class $P$ and explain why it is an important complexity class}

\subsection{Given two examples of problems and their natural encodings. Suggest other encodings that are not reasonable and explain why}

\subsection{Give an example of a problem which is in NP and prove its membership NP using the notion of certificate}

\subsection{We have given two definitions of the class NP. One uses ”certificates” and one uses nondeterministic Turing machines. Recall the two definitions and prove that they are equivalent}

\subsection{Explain why NP $\subseteq$ ExpTime}

\subsection{Define the notion of NP complete problem and explain why this notion is important}

\subsection{Define the notion of "polynomial time mapping reducible". Show that if $A$ is polynomial time reducible to $B$ and $B \in P$ then $A \in P$}

\subsection{Prove that 3SAT is polynomial time reducible to CLIQUE}

\subsection{Prove the following two statements: ($i$) if $B \in$ NP-Complete and $B \in$ P, then P = NP, ($ii$) if $B \in$ NP-Complete and $B \leq_{P} C$, and $C \in$ NP then C is NP-Complete}

\subsection{Give the main constructions and arguments underlying the proof of the Cook-Levin Theorem that establishes that SAT is NP-Complete}

\subsection{Prove that 3SAT $\leq_P$ VERTEX-COVER}

\subsection{Define the HAMILTONIAN-PATH problem and prove it is NP-complete}

\subsection{Give a dynamic programming algorithm for the PARTITION problem and explain why it does not imply P = NP}

\subsection{Give a dynamic programming algorithm for the SUBSET-SUM problem and explain why it does not imply P = NP}



\end{document}